
    




    
\documentclass[11pt]{article}

    
    \usepackage[breakable]{tcolorbox}
    \tcbset{nobeforeafter} % prevents tcolorboxes being placing in paragraphs
    \usepackage{float}
    \floatplacement{figure}{H} % forces figures to be placed at the correct location
    
    \usepackage[T1]{fontenc}
    % Nicer default font (+ math font) than Computer Modern for most use cases
    \usepackage{mathpazo}

    % Basic figure setup, for now with no caption control since it's done
    % automatically by Pandoc (which extracts ![](path) syntax from Markdown).
    \usepackage{graphicx}
    % We will generate all images so they have a width \maxwidth. This means
    % that they will get their normal width if they fit onto the page, but
    % are scaled down if they would overflow the margins.
    \makeatletter
    \def\maxwidth{\ifdim\Gin@nat@width>\linewidth\linewidth
    \else\Gin@nat@width\fi}
    \makeatother
    \let\Oldincludegraphics\includegraphics
    % Set max figure width to be 80% of text width, for now hardcoded.
    \renewcommand{\includegraphics}[1]{\Oldincludegraphics[width=.8\maxwidth]{#1}}
    % Ensure that by default, figures have no caption (until we provide a
    % proper Figure object with a Caption API and a way to capture that
    % in the conversion process - todo).
    \usepackage{caption}
    \DeclareCaptionLabelFormat{nolabel}{}
    \captionsetup{labelformat=nolabel}

    \usepackage{adjustbox} % Used to constrain images to a maximum size 
    \usepackage{xcolor} % Allow colors to be defined
    \usepackage{enumerate} % Needed for markdown enumerations to work
    \usepackage{geometry} % Used to adjust the document margins
    \usepackage{amsmath} % Equations
    \usepackage{amssymb} % Equations
    \usepackage{textcomp} % defines textquotesingle
    % Hack from http://tex.stackexchange.com/a/47451/13684:
    \AtBeginDocument{%
        \def\PYZsq{\textquotesingle}% Upright quotes in Pygmentized code
    }
    \usepackage{upquote} % Upright quotes for verbatim code
    \usepackage{eurosym} % defines \euro
    \usepackage[mathletters]{ucs} % Extended unicode (utf-8) support
    \usepackage[utf8x]{inputenc} % Allow utf-8 characters in the tex document
    \usepackage{fancyvrb} % verbatim replacement that allows latex
    \usepackage{grffile} % extends the file name processing of package graphics 
                         % to support a larger range 
    % The hyperref package gives us a pdf with properly built
    % internal navigation ('pdf bookmarks' for the table of contents,
    % internal cross-reference links, web links for URLs, etc.)
    \usepackage{hyperref}
    \usepackage{longtable} % longtable support required by pandoc >1.10
    \usepackage{booktabs}  % table support for pandoc > 1.12.2
    \usepackage[inline]{enumitem} % IRkernel/repr support (it uses the enumerate* environment)
    \usepackage[normalem]{ulem} % ulem is needed to support strikethroughs (\sout)
                                % normalem makes italics be italics, not underlines
    \usepackage{mathrsfs}
    

    
    % Colors for the hyperref package
    \definecolor{urlcolor}{rgb}{0,.145,.698}
    \definecolor{linkcolor}{rgb}{.71,0.21,0.01}
    \definecolor{citecolor}{rgb}{.12,.54,.11}

    % ANSI colors
    \definecolor{ansi-black}{HTML}{3E424D}
    \definecolor{ansi-black-intense}{HTML}{282C36}
    \definecolor{ansi-red}{HTML}{E75C58}
    \definecolor{ansi-red-intense}{HTML}{B22B31}
    \definecolor{ansi-green}{HTML}{00A250}
    \definecolor{ansi-green-intense}{HTML}{007427}
    \definecolor{ansi-yellow}{HTML}{DDB62B}
    \definecolor{ansi-yellow-intense}{HTML}{B27D12}
    \definecolor{ansi-blue}{HTML}{208FFB}
    \definecolor{ansi-blue-intense}{HTML}{0065CA}
    \definecolor{ansi-magenta}{HTML}{D160C4}
    \definecolor{ansi-magenta-intense}{HTML}{A03196}
    \definecolor{ansi-cyan}{HTML}{60C6C8}
    \definecolor{ansi-cyan-intense}{HTML}{258F8F}
    \definecolor{ansi-white}{HTML}{C5C1B4}
    \definecolor{ansi-white-intense}{HTML}{A1A6B2}
    \definecolor{ansi-default-inverse-fg}{HTML}{FFFFFF}
    \definecolor{ansi-default-inverse-bg}{HTML}{000000}

    % commands and environments needed by pandoc snippets
    % extracted from the output of `pandoc -s`
    \providecommand{\tightlist}{%
      \setlength{\itemsep}{0pt}\setlength{\parskip}{0pt}}
    \DefineVerbatimEnvironment{Highlighting}{Verbatim}{commandchars=\\\{\}}
    % Add ',fontsize=\small' for more characters per line
    \newenvironment{Shaded}{}{}
    \newcommand{\KeywordTok}[1]{\textcolor[rgb]{0.00,0.44,0.13}{\textbf{{#1}}}}
    \newcommand{\DataTypeTok}[1]{\textcolor[rgb]{0.56,0.13,0.00}{{#1}}}
    \newcommand{\DecValTok}[1]{\textcolor[rgb]{0.25,0.63,0.44}{{#1}}}
    \newcommand{\BaseNTok}[1]{\textcolor[rgb]{0.25,0.63,0.44}{{#1}}}
    \newcommand{\FloatTok}[1]{\textcolor[rgb]{0.25,0.63,0.44}{{#1}}}
    \newcommand{\CharTok}[1]{\textcolor[rgb]{0.25,0.44,0.63}{{#1}}}
    \newcommand{\StringTok}[1]{\textcolor[rgb]{0.25,0.44,0.63}{{#1}}}
    \newcommand{\CommentTok}[1]{\textcolor[rgb]{0.38,0.63,0.69}{\textit{{#1}}}}
    \newcommand{\OtherTok}[1]{\textcolor[rgb]{0.00,0.44,0.13}{{#1}}}
    \newcommand{\AlertTok}[1]{\textcolor[rgb]{1.00,0.00,0.00}{\textbf{{#1}}}}
    \newcommand{\FunctionTok}[1]{\textcolor[rgb]{0.02,0.16,0.49}{{#1}}}
    \newcommand{\RegionMarkerTok}[1]{{#1}}
    \newcommand{\ErrorTok}[1]{\textcolor[rgb]{1.00,0.00,0.00}{\textbf{{#1}}}}
    \newcommand{\NormalTok}[1]{{#1}}
    
    % Additional commands for more recent versions of Pandoc
    \newcommand{\ConstantTok}[1]{\textcolor[rgb]{0.53,0.00,0.00}{{#1}}}
    \newcommand{\SpecialCharTok}[1]{\textcolor[rgb]{0.25,0.44,0.63}{{#1}}}
    \newcommand{\VerbatimStringTok}[1]{\textcolor[rgb]{0.25,0.44,0.63}{{#1}}}
    \newcommand{\SpecialStringTok}[1]{\textcolor[rgb]{0.73,0.40,0.53}{{#1}}}
    \newcommand{\ImportTok}[1]{{#1}}
    \newcommand{\DocumentationTok}[1]{\textcolor[rgb]{0.73,0.13,0.13}{\textit{{#1}}}}
    \newcommand{\AnnotationTok}[1]{\textcolor[rgb]{0.38,0.63,0.69}{\textbf{\textit{{#1}}}}}
    \newcommand{\CommentVarTok}[1]{\textcolor[rgb]{0.38,0.63,0.69}{\textbf{\textit{{#1}}}}}
    \newcommand{\VariableTok}[1]{\textcolor[rgb]{0.10,0.09,0.49}{{#1}}}
    \newcommand{\ControlFlowTok}[1]{\textcolor[rgb]{0.00,0.44,0.13}{\textbf{{#1}}}}
    \newcommand{\OperatorTok}[1]{\textcolor[rgb]{0.40,0.40,0.40}{{#1}}}
    \newcommand{\BuiltInTok}[1]{{#1}}
    \newcommand{\ExtensionTok}[1]{{#1}}
    \newcommand{\PreprocessorTok}[1]{\textcolor[rgb]{0.74,0.48,0.00}{{#1}}}
    \newcommand{\AttributeTok}[1]{\textcolor[rgb]{0.49,0.56,0.16}{{#1}}}
    \newcommand{\InformationTok}[1]{\textcolor[rgb]{0.38,0.63,0.69}{\textbf{\textit{{#1}}}}}
    \newcommand{\WarningTok}[1]{\textcolor[rgb]{0.38,0.63,0.69}{\textbf{\textit{{#1}}}}}
    
    
    % Define a nice break command that doesn't care if a line doesn't already
    % exist.
    \def\br{\hspace*{\fill} \\* }
    % Math Jax compatibility definitions
    \def\gt{>}
    \def\lt{<}
    \let\Oldtex\TeX
    \let\Oldlatex\LaTeX
    \renewcommand{\TeX}{\textrm{\Oldtex}}
    \renewcommand{\LaTeX}{\textrm{\Oldlatex}}
    % Document parameters
    % Document title
    \title{numerical\_methods}
    
    
    
    
    
% Pygments definitions
\makeatletter
\def\PY@reset{\let\PY@it=\relax \let\PY@bf=\relax%
    \let\PY@ul=\relax \let\PY@tc=\relax%
    \let\PY@bc=\relax \let\PY@ff=\relax}
\def\PY@tok#1{\csname PY@tok@#1\endcsname}
\def\PY@toks#1+{\ifx\relax#1\empty\else%
    \PY@tok{#1}\expandafter\PY@toks\fi}
\def\PY@do#1{\PY@bc{\PY@tc{\PY@ul{%
    \PY@it{\PY@bf{\PY@ff{#1}}}}}}}
\def\PY#1#2{\PY@reset\PY@toks#1+\relax+\PY@do{#2}}

\@namedef{PY@tok@w}{\def\PY@tc##1{\textcolor[rgb]{0.73,0.73,0.73}{##1}}}
\@namedef{PY@tok@c}{\let\PY@it=\textit\def\PY@tc##1{\textcolor[rgb]{0.25,0.50,0.50}{##1}}}
\@namedef{PY@tok@cp}{\def\PY@tc##1{\textcolor[rgb]{0.74,0.48,0.00}{##1}}}
\@namedef{PY@tok@k}{\let\PY@bf=\textbf\def\PY@tc##1{\textcolor[rgb]{0.00,0.50,0.00}{##1}}}
\@namedef{PY@tok@kp}{\def\PY@tc##1{\textcolor[rgb]{0.00,0.50,0.00}{##1}}}
\@namedef{PY@tok@kt}{\def\PY@tc##1{\textcolor[rgb]{0.69,0.00,0.25}{##1}}}
\@namedef{PY@tok@o}{\def\PY@tc##1{\textcolor[rgb]{0.40,0.40,0.40}{##1}}}
\@namedef{PY@tok@ow}{\let\PY@bf=\textbf\def\PY@tc##1{\textcolor[rgb]{0.67,0.13,1.00}{##1}}}
\@namedef{PY@tok@nb}{\def\PY@tc##1{\textcolor[rgb]{0.00,0.50,0.00}{##1}}}
\@namedef{PY@tok@nf}{\def\PY@tc##1{\textcolor[rgb]{0.00,0.00,1.00}{##1}}}
\@namedef{PY@tok@nc}{\let\PY@bf=\textbf\def\PY@tc##1{\textcolor[rgb]{0.00,0.00,1.00}{##1}}}
\@namedef{PY@tok@nn}{\let\PY@bf=\textbf\def\PY@tc##1{\textcolor[rgb]{0.00,0.00,1.00}{##1}}}
\@namedef{PY@tok@ne}{\let\PY@bf=\textbf\def\PY@tc##1{\textcolor[rgb]{0.82,0.25,0.23}{##1}}}
\@namedef{PY@tok@nv}{\def\PY@tc##1{\textcolor[rgb]{0.10,0.09,0.49}{##1}}}
\@namedef{PY@tok@no}{\def\PY@tc##1{\textcolor[rgb]{0.53,0.00,0.00}{##1}}}
\@namedef{PY@tok@nl}{\def\PY@tc##1{\textcolor[rgb]{0.63,0.63,0.00}{##1}}}
\@namedef{PY@tok@ni}{\let\PY@bf=\textbf\def\PY@tc##1{\textcolor[rgb]{0.60,0.60,0.60}{##1}}}
\@namedef{PY@tok@na}{\def\PY@tc##1{\textcolor[rgb]{0.49,0.56,0.16}{##1}}}
\@namedef{PY@tok@nt}{\let\PY@bf=\textbf\def\PY@tc##1{\textcolor[rgb]{0.00,0.50,0.00}{##1}}}
\@namedef{PY@tok@nd}{\def\PY@tc##1{\textcolor[rgb]{0.67,0.13,1.00}{##1}}}
\@namedef{PY@tok@s}{\def\PY@tc##1{\textcolor[rgb]{0.73,0.13,0.13}{##1}}}
\@namedef{PY@tok@sd}{\let\PY@it=\textit\def\PY@tc##1{\textcolor[rgb]{0.73,0.13,0.13}{##1}}}
\@namedef{PY@tok@si}{\let\PY@bf=\textbf\def\PY@tc##1{\textcolor[rgb]{0.73,0.40,0.53}{##1}}}
\@namedef{PY@tok@se}{\let\PY@bf=\textbf\def\PY@tc##1{\textcolor[rgb]{0.73,0.40,0.13}{##1}}}
\@namedef{PY@tok@sr}{\def\PY@tc##1{\textcolor[rgb]{0.73,0.40,0.53}{##1}}}
\@namedef{PY@tok@ss}{\def\PY@tc##1{\textcolor[rgb]{0.10,0.09,0.49}{##1}}}
\@namedef{PY@tok@sx}{\def\PY@tc##1{\textcolor[rgb]{0.00,0.50,0.00}{##1}}}
\@namedef{PY@tok@m}{\def\PY@tc##1{\textcolor[rgb]{0.40,0.40,0.40}{##1}}}
\@namedef{PY@tok@gh}{\let\PY@bf=\textbf\def\PY@tc##1{\textcolor[rgb]{0.00,0.00,0.50}{##1}}}
\@namedef{PY@tok@gu}{\let\PY@bf=\textbf\def\PY@tc##1{\textcolor[rgb]{0.50,0.00,0.50}{##1}}}
\@namedef{PY@tok@gd}{\def\PY@tc##1{\textcolor[rgb]{0.63,0.00,0.00}{##1}}}
\@namedef{PY@tok@gi}{\def\PY@tc##1{\textcolor[rgb]{0.00,0.63,0.00}{##1}}}
\@namedef{PY@tok@gr}{\def\PY@tc##1{\textcolor[rgb]{1.00,0.00,0.00}{##1}}}
\@namedef{PY@tok@ge}{\let\PY@it=\textit}
\@namedef{PY@tok@gs}{\let\PY@bf=\textbf}
\@namedef{PY@tok@gp}{\let\PY@bf=\textbf\def\PY@tc##1{\textcolor[rgb]{0.00,0.00,0.50}{##1}}}
\@namedef{PY@tok@go}{\def\PY@tc##1{\textcolor[rgb]{0.53,0.53,0.53}{##1}}}
\@namedef{PY@tok@gt}{\def\PY@tc##1{\textcolor[rgb]{0.00,0.27,0.87}{##1}}}
\@namedef{PY@tok@err}{\def\PY@bc##1{{\setlength{\fboxsep}{\string -\fboxrule}\fcolorbox[rgb]{1.00,0.00,0.00}{1,1,1}{\strut ##1}}}}
\@namedef{PY@tok@kc}{\let\PY@bf=\textbf\def\PY@tc##1{\textcolor[rgb]{0.00,0.50,0.00}{##1}}}
\@namedef{PY@tok@kd}{\let\PY@bf=\textbf\def\PY@tc##1{\textcolor[rgb]{0.00,0.50,0.00}{##1}}}
\@namedef{PY@tok@kn}{\let\PY@bf=\textbf\def\PY@tc##1{\textcolor[rgb]{0.00,0.50,0.00}{##1}}}
\@namedef{PY@tok@kr}{\let\PY@bf=\textbf\def\PY@tc##1{\textcolor[rgb]{0.00,0.50,0.00}{##1}}}
\@namedef{PY@tok@bp}{\def\PY@tc##1{\textcolor[rgb]{0.00,0.50,0.00}{##1}}}
\@namedef{PY@tok@fm}{\def\PY@tc##1{\textcolor[rgb]{0.00,0.00,1.00}{##1}}}
\@namedef{PY@tok@vc}{\def\PY@tc##1{\textcolor[rgb]{0.10,0.09,0.49}{##1}}}
\@namedef{PY@tok@vg}{\def\PY@tc##1{\textcolor[rgb]{0.10,0.09,0.49}{##1}}}
\@namedef{PY@tok@vi}{\def\PY@tc##1{\textcolor[rgb]{0.10,0.09,0.49}{##1}}}
\@namedef{PY@tok@vm}{\def\PY@tc##1{\textcolor[rgb]{0.10,0.09,0.49}{##1}}}
\@namedef{PY@tok@sa}{\def\PY@tc##1{\textcolor[rgb]{0.73,0.13,0.13}{##1}}}
\@namedef{PY@tok@sb}{\def\PY@tc##1{\textcolor[rgb]{0.73,0.13,0.13}{##1}}}
\@namedef{PY@tok@sc}{\def\PY@tc##1{\textcolor[rgb]{0.73,0.13,0.13}{##1}}}
\@namedef{PY@tok@dl}{\def\PY@tc##1{\textcolor[rgb]{0.73,0.13,0.13}{##1}}}
\@namedef{PY@tok@s2}{\def\PY@tc##1{\textcolor[rgb]{0.73,0.13,0.13}{##1}}}
\@namedef{PY@tok@sh}{\def\PY@tc##1{\textcolor[rgb]{0.73,0.13,0.13}{##1}}}
\@namedef{PY@tok@s1}{\def\PY@tc##1{\textcolor[rgb]{0.73,0.13,0.13}{##1}}}
\@namedef{PY@tok@mb}{\def\PY@tc##1{\textcolor[rgb]{0.40,0.40,0.40}{##1}}}
\@namedef{PY@tok@mf}{\def\PY@tc##1{\textcolor[rgb]{0.40,0.40,0.40}{##1}}}
\@namedef{PY@tok@mh}{\def\PY@tc##1{\textcolor[rgb]{0.40,0.40,0.40}{##1}}}
\@namedef{PY@tok@mi}{\def\PY@tc##1{\textcolor[rgb]{0.40,0.40,0.40}{##1}}}
\@namedef{PY@tok@il}{\def\PY@tc##1{\textcolor[rgb]{0.40,0.40,0.40}{##1}}}
\@namedef{PY@tok@mo}{\def\PY@tc##1{\textcolor[rgb]{0.40,0.40,0.40}{##1}}}
\@namedef{PY@tok@ch}{\let\PY@it=\textit\def\PY@tc##1{\textcolor[rgb]{0.25,0.50,0.50}{##1}}}
\@namedef{PY@tok@cm}{\let\PY@it=\textit\def\PY@tc##1{\textcolor[rgb]{0.25,0.50,0.50}{##1}}}
\@namedef{PY@tok@cpf}{\let\PY@it=\textit\def\PY@tc##1{\textcolor[rgb]{0.25,0.50,0.50}{##1}}}
\@namedef{PY@tok@c1}{\let\PY@it=\textit\def\PY@tc##1{\textcolor[rgb]{0.25,0.50,0.50}{##1}}}
\@namedef{PY@tok@cs}{\let\PY@it=\textit\def\PY@tc##1{\textcolor[rgb]{0.25,0.50,0.50}{##1}}}

\def\PYZbs{\char`\\}
\def\PYZus{\char`\_}
\def\PYZob{\char`\{}
\def\PYZcb{\char`\}}
\def\PYZca{\char`\^}
\def\PYZam{\char`\&}
\def\PYZlt{\char`\<}
\def\PYZgt{\char`\>}
\def\PYZsh{\char`\#}
\def\PYZpc{\char`\%}
\def\PYZdl{\char`\$}
\def\PYZhy{\char`\-}
\def\PYZsq{\char`\'}
\def\PYZdq{\char`\"}
\def\PYZti{\char`\~}
% for compatibility with earlier versions
\def\PYZat{@}
\def\PYZlb{[}
\def\PYZrb{]}
\makeatother


    % For linebreaks inside Verbatim environment from package fancyvrb. 
    \makeatletter
        \newbox\Wrappedcontinuationbox 
        \newbox\Wrappedvisiblespacebox 
        \newcommand*\Wrappedvisiblespace {\textcolor{red}{\textvisiblespace}} 
        \newcommand*\Wrappedcontinuationsymbol {\textcolor{red}{\llap{\tiny$\m@th\hookrightarrow$}}} 
        \newcommand*\Wrappedcontinuationindent {3ex } 
        \newcommand*\Wrappedafterbreak {\kern\Wrappedcontinuationindent\copy\Wrappedcontinuationbox} 
        % Take advantage of the already applied Pygments mark-up to insert 
        % potential linebreaks for TeX processing. 
        %        {, <, #, %, $, ' and ": go to next line. 
        %        _, }, ^, &, >, - and ~: stay at end of broken line. 
        % Use of \textquotesingle for straight quote. 
        \newcommand*\Wrappedbreaksatspecials {% 
            \def\PYGZus{\discretionary{\char`\_}{\Wrappedafterbreak}{\char`\_}}% 
            \def\PYGZob{\discretionary{}{\Wrappedafterbreak\char`\{}{\char`\{}}% 
            \def\PYGZcb{\discretionary{\char`\}}{\Wrappedafterbreak}{\char`\}}}% 
            \def\PYGZca{\discretionary{\char`\^}{\Wrappedafterbreak}{\char`\^}}% 
            \def\PYGZam{\discretionary{\char`\&}{\Wrappedafterbreak}{\char`\&}}% 
            \def\PYGZlt{\discretionary{}{\Wrappedafterbreak\char`\<}{\char`\<}}% 
            \def\PYGZgt{\discretionary{\char`\>}{\Wrappedafterbreak}{\char`\>}}% 
            \def\PYGZsh{\discretionary{}{\Wrappedafterbreak\char`\#}{\char`\#}}% 
            \def\PYGZpc{\discretionary{}{\Wrappedafterbreak\char`\%}{\char`\%}}% 
            \def\PYGZdl{\discretionary{}{\Wrappedafterbreak\char`\$}{\char`\$}}% 
            \def\PYGZhy{\discretionary{\char`\-}{\Wrappedafterbreak}{\char`\-}}% 
            \def\PYGZsq{\discretionary{}{\Wrappedafterbreak\textquotesingle}{\textquotesingle}}% 
            \def\PYGZdq{\discretionary{}{\Wrappedafterbreak\char`\"}{\char`\"}}% 
            \def\PYGZti{\discretionary{\char`\~}{\Wrappedafterbreak}{\char`\~}}% 
        } 
        % Some characters . , ; ? ! / are not pygmentized. 
        % This macro makes them "active" and they will insert potential linebreaks 
        \newcommand*\Wrappedbreaksatpunct {% 
            \lccode`\~`\.\lowercase{\def~}{\discretionary{\hbox{\char`\.}}{\Wrappedafterbreak}{\hbox{\char`\.}}}% 
            \lccode`\~`\,\lowercase{\def~}{\discretionary{\hbox{\char`\,}}{\Wrappedafterbreak}{\hbox{\char`\,}}}% 
            \lccode`\~`\;\lowercase{\def~}{\discretionary{\hbox{\char`\;}}{\Wrappedafterbreak}{\hbox{\char`\;}}}% 
            \lccode`\~`\:\lowercase{\def~}{\discretionary{\hbox{\char`\:}}{\Wrappedafterbreak}{\hbox{\char`\:}}}% 
            \lccode`\~`\?\lowercase{\def~}{\discretionary{\hbox{\char`\?}}{\Wrappedafterbreak}{\hbox{\char`\?}}}% 
            \lccode`\~`\!\lowercase{\def~}{\discretionary{\hbox{\char`\!}}{\Wrappedafterbreak}{\hbox{\char`\!}}}% 
            \lccode`\~`\/\lowercase{\def~}{\discretionary{\hbox{\char`\/}}{\Wrappedafterbreak}{\hbox{\char`\/}}}% 
            \catcode`\.\active
            \catcode`\,\active 
            \catcode`\;\active
            \catcode`\:\active
            \catcode`\?\active
            \catcode`\!\active
            \catcode`\/\active 
            \lccode`\~`\~ 	
        }
    \makeatother

    \let\OriginalVerbatim=\Verbatim
    \makeatletter
    \renewcommand{\Verbatim}[1][1]{%
        %\parskip\z@skip
        \sbox\Wrappedcontinuationbox {\Wrappedcontinuationsymbol}%
        \sbox\Wrappedvisiblespacebox {\FV@SetupFont\Wrappedvisiblespace}%
        \def\FancyVerbFormatLine ##1{\hsize\linewidth
            \vtop{\raggedright\hyphenpenalty\z@\exhyphenpenalty\z@
                \doublehyphendemerits\z@\finalhyphendemerits\z@
                \strut ##1\strut}%
        }%
        % If the linebreak is at a space, the latter will be displayed as visible
        % space at end of first line, and a continuation symbol starts next line.
        % Stretch/shrink are however usually zero for typewriter font.
        \def\FV@Space {%
            \nobreak\hskip\z@ plus\fontdimen3\font minus\fontdimen4\font
            \discretionary{\copy\Wrappedvisiblespacebox}{\Wrappedafterbreak}
            {\kern\fontdimen2\font}%
        }%
        
        % Allow breaks at special characters using \PYG... macros.
        \Wrappedbreaksatspecials
        % Breaks at punctuation characters . , ; ? ! and / need catcode=\active 	
        \OriginalVerbatim[#1,codes*=\Wrappedbreaksatpunct]%
    }
    \makeatother

    % Exact colors from NB
    \definecolor{incolor}{HTML}{303F9F}
    \definecolor{outcolor}{HTML}{D84315}
    \definecolor{cellborder}{HTML}{CFCFCF}
    \definecolor{cellbackground}{HTML}{F7F7F7}
    
    % prompt
    \newcommand{\prompt}[4]{
        \llap{{\color{#2}[#3]: #4}}\vspace{-1.25em}
    }
    

    
    % Prevent overflowing lines due to hard-to-break entities
    \sloppy 
    % Setup hyperref package
    \hypersetup{
      breaklinks=true,  % so long urls are correctly broken across lines
      colorlinks=true,
      urlcolor=urlcolor,
      linkcolor=linkcolor,
      citecolor=citecolor,
      }
    % Slightly bigger margins than the latex defaults
    
    \geometry{verbose,tmargin=1in,bmargin=1in,lmargin=1in,rmargin=1in}
    
    

    \begin{document}
    
    
    \maketitle
    
    

    
    \hypertarget{python-expresivo}{%
\section{Python expresivo}\label{python-expresivo}}

No se busca que esta sea una introducción a la programación. Tutoriales
sobre el lenguaje de programación Python se pueden encontrar en Internet
fácilmente. Lo que se busca hacer en este capítulo es proveer una rápida
recopilación de las características de Python.

\hypertarget{por-quuxe9-python}{%
\subsection{¿Por qué Python?}\label{por-quuxe9-python}}

En esta guía se usa Python 3, con la idea de facilitar la aplicación de
algunos de los temas que se discutirán. Con su sintaxis razonablemente
simple se pueden escribir programas expresivamente, aunque a veces se
sacrifique eficiencia. Las implementaciones tienen un objetivo
pedagógico. Cuando se requiera más poder computacional, puede que se
necesite cambiar a un lenguaje compilado. La mayoría del trabajo en
supercomputadoras se realiza usando lenguajes como Fortran o C++. Sin
embargo, construir un programa prototipo en Python puede ser invaluable
para adquirir una mayor comprensión.

\hypertarget{calidad-del-cuxf3digo}{%
\subsection{Calidad del código}\label{calidad-del-cuxf3digo}}

Cuando se desarrollan programas grandes, seguir una guía sobre el
formato que debe tener el código puede ser importante, pero la mayoría
de los programas mostrados aquí serán cortos por lo que no será un tema
importante. A veces, preguntas sobre cómo escribir y revisar programas
son más importantes que el formato del código. Algunos consejos
generales son los siguientes:

\begin{itemize}
\tightlist
\item
  \textbf{La legibilidad del código es importante} Asegúrese de que sus
  programas estén dirigidos a humanos, no computadoras. Esto significa
  evitar usar trucos ``astutos''. Por lo que debe usar buenos nombres de
  variables y escribir comentarios que agreguen valor (en lugar de
  repetir el código). El mayor beneficiado será usted, cuando regrese a
  sus programas meses después.
\item
  \textbf{Sea cuidadoso, no rápido, cuando escriba código} Generalmente
  es más difícil debuggear código que escribirlo. En lugar de pasar dos
  minutos escribiendo un programa que no sirve y requiera dos horas para
  arreglar, intente pasar 10 minutos diseñando el código, siendo
  cuidadoso de convertir sus ideas a líneas de un programa. Tampoco hace
  daño usar Python de forma interactiva para probar componentes del
  código una a una o diferentes partes en conjunto.
\item
  \textbf{Código sin probar es mal código} Asegúrese que su código
  funciona correctamente. Si tiene un ejemplo del que ya conoce la
  respuesta, asegúrese de que su código devuelva esa respuesta. Siga
  manualmente una variedad de casos (mentalmente o en papel). Esto y
  usando \emph{print statements} puede ayudar mucho para asegurar que
  todo funcione como se debe.
\item
  \textbf{Escriba funciones que hagan una cosa bien} En lugar de
  escribir muchas operaciones no relacionadas en secuencia, debe
  estructurar su código para que haga uso de funciones bien nombradas (y
  bien pensadas) que hagan una cosa bien.
\item
  \textbf{Use librerías confiables} En la mayoría de esta guía se
  ``reinventará la rueda'', con el fin de entender cómo funcionan las
  cosas. Tiempo después en su trabajo, usted no siempre debe usar sus
  propias implementaciones de algoritmos estándar.
\end{itemize}

\hypertarget{resumen-de-caracteruxedsticas-de-python}{%
\section{Resumen de características de
Python}\label{resumen-de-caracteruxedsticas-de-python}}

\hypertarget{lo-buxe1sico}{%
\subsection{Lo básico}\label{lo-buxe1sico}}

Python se puede usar de forma interactiva: cuando vea el comando
\texttt{\textgreater{}\textgreater{}\textgreater{}}, también conocido
como \emph{chevron}. Pero también se puede usar como otros lenguajes,
donde se almacena el código en un archivo.\\
Como otros lenguajes, en Python se usan variables, que pueden ser
números enteros, complejos, etc. A diferencia de otros lenguajes, Python
es dinámico en cuanto a los tipos de variable. Por ejemplo,
\texttt{x\ =\ 0.5} crea una variable de punto flotante. Números como
\texttt{0.5} o strings como \texttt{"Hello"} se conocen como
\emph{literales}. Si se desea imprimir en pantalla una variable, se usa
la función \texttt{print(x)} que viene predefinida. Funciones
adicionales están disponibles a través de la librería estándar, por
ejemplo usted puede hacer un \texttt{import} de la librería
\texttt{math} para hacer uso de la función \texttt{sqrt}.\\
Se pueden realizar operaciones aritméticas con las variables.
\texttt{x**y} eleva la variable \texttt{x} a la potencia \texttt{y} ó
\texttt{x//y} hace una \emph{floor division} (hace la división y
redondea el resultado hacia abajo).\\
Python también utiliza asignación aumentada, es decir, operaciones como
\texttt{x\ +=\ 1} o incluso asignaciones múltiples como
\texttt{x,\ y\ =\ 0.5,\ "Hello"}.\\
Los comentarios son una característica importante de los lenguajes de
programación: son texto ignorado por la computadora pero muy útil para
los humanos que leen el código. Ese humano puede ser usted en unos
meses, cuando se haya olvidado el propósito o los detalles del código
que esté inspeccionando. Python permite escribir comentarios tanto de
una línea como de múltiples líneas, usando \texttt{\#} ó
\emph{docstrings} a través de tres comillas \texttt{"""}.

\hypertarget{control-de-flujo}{%
\subsection{Control de flujo}\label{control-de-flujo}}

Control de flujo se refiere a construcciones programables donde no todas
las líneas del código son ejecutadas en orden. Un ejemplo es la
declaración con \texttt{if}

    \begin{tcolorbox}[breakable, size=fbox, boxrule=1pt, pad at break*=1mm,colback=cellbackground, colframe=cellborder]
\prompt{In}{incolor}{1}{\hspace{4pt}}
\begin{Verbatim}[commandchars=\\\{\}]
\PY{n}{x} \PY{o}{=} \PY{l+m+mf}{0.5}
\PY{k}{if} \PY{n}{x} \PY{o}{!=} \PY{l+m+mi}{0}\PY{p}{:}
    \PY{n+nb}{print}\PY{p}{(}\PY{l+s+s2}{\PYZdq{}}\PY{l+s+s2}{x no es cero}\PY{l+s+s2}{\PYZdq{}}\PY{p}{)}
\end{Verbatim}
\end{tcolorbox}

    \begin{Verbatim}[commandchars=\\\{\}]
x no es cero
\end{Verbatim}

    La sangría o \emph{tabulación} es importante en Python: la línea después
del \texttt{if} está tabulada, dando a entender que pertenece al
escenario correspondiente. Similarmente, los dos puntos, \texttt{:}, al
final de la línea que contiene al \texttt{if} es sintácticamente
importante. Si se quiere tomar en cuenta otras posibilidades, se puede
usar otro bloque tabulado empezando con un \texttt{else:} o con
\texttt{elif\ x==0:}.\\
Otra construcción programable es ciclo \texttt{for}, que se usa cuando
se desea repetir una cierta acción un número fijo de veces. Por ejemplo,
escribiendo \texttt{for\ i\ in\ range(3):} se repite lo que sea que siga
tres veces.

\hypertarget{estructuras-de-datos}{%
\subsection{Estructuras de datos}\label{estructuras-de-datos}}

Python utiliza entidades contenedoras, llamadas estructuras de datos.

\begin{itemize}
\tightlist
\item
  \textbf{Listas} Una lista es un contenedor de elementos. Puede
  aumentar cuando se necesite y los elementos pueden ser de diferentes
  tipos. Para crear una lista se usan corchetes, y los elementos se
  pueden separar por comas, por ejemplo
  \texttt{z\ =\ {[}3,\ 1+2j,\ -2.0{]}}
\item
  \textbf{Tuplas} Las tuplas se pueden definir como listas inmutables.
  Son secuencias que no cambian ni crecen. Son definidas usando
  paréntesis en lugar de corchetes: \texttt{xs\ =\ (1,2,3)}, pero se
  pueden omitir los paréntesis.
\item
  \textbf{Strings} Strings pueden verse como secuencias. Si
  \texttt{name\ =\ "Mary"} entonces \texttt{name{[}-1} es la letra `y'.
  Como las tuplas, las strings son inmutables, y se pueden concatenar
  dos strings usando un \texttt{+}.
\item
  \textbf{Diccionarios} Python también soporta diccionarios, que son
  llamados listas asociativas en computación. Su información se puede
  acceder usando strings o \emph{floats} (números de punto flotante)
  como \emph{llaves}. En otras palabras, diccionarios contienen pares de
  llave y valor. La sintaxis para crear uno utiliza braquets, con pares
  llave-valor separados por dos puntos. Por ejemplo,
  \texttt{htow\ =\ \{1.41:\ 31.3,\ 1.45:\ 36.7,\ 1.48:\ 42.4\}} es un
  diccionario que relaciona alturas con pesos. Para acceder a un valor,
  se utilizan corchetes con la llave correspondiente:
  \texttt{htow{[}1.45{]}}.
\end{itemize}

\hypertarget{funciones-definidas-por-el-usuario}{%
\subsection{Funciones definidas por el
usuario}\label{funciones-definidas-por-el-usuario}}

Si en un programa se ejecutan muchas operaciones en secuencia, dentro de
varios ciclos, su lógica se puede volver confusa. Por lo tanto, se
pueden agrupar operaciones lógicamente relacionadas y crear lo que se
llama funciones definidas por el usuario. Estas se refieren a líneas de
código que no necesariamente se ejecutan en el orden en el que aparecen
en el código.\\
Para introducir una función, se utiliza la palabra \texttt{def}, junto
con un nombre y dos puntos al final de la línea, así como tabulaciones
dentro del bloque de código que sigue. Por ejemplo, esta es una función
que suma desde 1 hasta algún entero:

    \begin{tcolorbox}[breakable, size=fbox, boxrule=1pt, pad at break*=1mm,colback=cellbackground, colframe=cellborder]
\prompt{In}{incolor}{2}{\hspace{4pt}}
\begin{Verbatim}[commandchars=\\\{\}]
\PY{k}{def} \PY{n+nf}{sumOfInts}\PY{p}{(}\PY{n}{nmax}\PY{p}{)}\PY{p}{:}
    \PY{n}{val} \PY{o}{=} \PY{n+nb}{sum}\PY{p}{(}\PY{n+nb}{range}\PY{p}{(}\PY{l+m+mi}{1}\PY{p}{,} \PY{n}{nmax} \PY{o}{+} \PY{l+m+mi}{1}\PY{p}{)}\PY{p}{)}
    \PY{k}{return} \PY{n}{val}
\end{Verbatim}
\end{tcolorbox}

    Esta función recibe un solo parámetro y devuelve un sólo valor. Pero
podría recibir varios argumentos, o ninguno. Y similarmente, pudo haber
impreso el resultado en pantalla, en lugar de devolverlo.

\hypertarget{errores}{%
\section{Errores}\label{errores}}

En este texto se usa la palabra \emph{exactitud} para describir el
acierto de un valor con su (posiblemente desconocido) valor verdadero.
Por otro lado, la palabra \emph{precisión} se usa para denotar cuántos
dígitos se pueden usar en una operación matemática, aunque sean
correctos o no. Un resultado inexacto se obtiene cuando hay un error.
Esto puede pasar por una variedad de razones, de las que solo una es
precisión limitada. Excluyendo el ``error humano'' e incertidumbre de
las mediciones en los datos entrantes, hay generalmente dos tipos de
errores con los que se trata en cómputo numérico: error de aproximación
y error de redondeo. En detalle:

\begin{itemize}
\tightlist
\item
  \textbf{Error de aproximación} Aquí un ejemplo. Usted está tratando de
  aproximar la exponencial, \(e^x\), usando su serie de Taylor:
\end{itemize}

\begin{equation}
y = \sum_{n=0}^{n_{max}} \frac{x^n}{n!}  \label{2.4} \tag{1}
\end{equation}

Obviamente, estamos limitando la suma a los términos hasta \(n_{max}\)
(es decir, estamos incluyendo los terminos marcados como 0, 1, \ldots,
\(n_{max}\) y no usamos los términos desde \(n_{max}\) hasta
\(\infty\)). Como resultado, es un tanto obvio que el valor de \(y\)
para un \(x\) dado puede depender de \(n_{max}\). En principio, al costo
de correr los cálculos por más tiempo, uno puede obtener una mejor
respuesta.

\begin{itemize}
\tightlist
\item
  \textbf{Errores de redondeo} Este tipo de errores aparecen cada vez
  que un cálculo es realizado sin usar números de punto flotante: como
  estos no tienen una infinita precisión, algo de información se pierde.
  Por ejemplo: usando números reales, es fácil ver que
  \((\sqrt{2})^2 - 2 = 0\). Sin embargo, cuando se realiza la misma
  operación en Python obtenemos una respuesta diferente de cero:
\end{itemize}

    \begin{tcolorbox}[breakable, size=fbox, boxrule=1pt, pad at break*=1mm,colback=cellbackground, colframe=cellborder]
\prompt{In}{incolor}{3}{\hspace{4pt}}
\begin{Verbatim}[commandchars=\\\{\}]
\PY{k+kn}{from} \PY{n+nn}{math} \PY{k+kn}{import} \PY{n}{sqrt}
\PY{p}{(}\PY{n}{sqrt}\PY{p}{(}\PY{l+m+mi}{2}\PY{p}{)}\PY{p}{)}\PY{o}{*}\PY{o}{*}\PY{l+m+mi}{2} \PY{o}{\PYZhy{}} \PY{l+m+mi}{2}
\end{Verbatim}
\end{tcolorbox}

            \begin{tcolorbox}[breakable, boxrule=.5pt, size=fbox, pad at break*=1mm, opacityfill=0]
\prompt{Out}{outcolor}{3}{\hspace{3.5pt}}
\begin{Verbatim}[commandchars=\\\{\}]
4.440892098500626e-16
\end{Verbatim}
\end{tcolorbox}
        
    Esto es porque \(\sqrt{2}\) no puede ser evaluado con una cantidad
infinita de dígitos en una computadora. Por lo tanto, el (un tanto
inexacto) resultado de \(\sqrt{2}\) es usado para realizar un segundo
cálculo: el cuadrado. Finalmente, la resta es aún otra operación
matemática que puede llevar a un error de redondeo. A menudo, los
errores de redondeo no desaparecen aunque se haga el cálculo por más
tiempo.\(\sqrt{2}\)

\hypertarget{error-absoluto-y-relativo}{%
\subsection{Error absoluto y relativo}\label{error-absoluto-y-relativo}}

Asuma que estamos estudiando una cantidad cuyo valor exacto es \(x\). Si
\(\tilde{x}\) es un valor aproximado para este valor, entonces se define
el \emph{error absoluto} como sigue:

\[
\begin{equation} 
\Delta x = \tilde{x} - x  \label{2.5}\tag{2}
\end{equation}
\]

Aquí no especificamos en este momento el origen de este error absoluto:
puede ser por incertesas en los datos de entrada, una inexactitud
introducida por un cálculo imperfecto como el mostrado anteriormente, o
el resultado de un error de redondeo (posiblemente acumulado sobre
varios cálculos). Por ejemplo:

\[
\begin{equation}
x_0 = 1.000, \quad \quad \tilde{x_0} = 0.999 \label{2.6}\tag{3}
\end{equation}
\]

Corresponde a un error absoluto de \(\Delta x_0 = -10^{-3}\) . Esto
también nos permite ver que el error absoluto, por definición, puede ser
positivo o negativo. Si se necesita que sea positivo (digamos, para
obtener su logaritmo), simplemente se toma su valor absoluto.\\
Usualmente estamos interesados en definir un \emph{límite de error} de
la forma:

\[
\begin{equation}
|\Delta x| \leq \epsilon \label{2.7}\tag{4}
\end{equation}
\]

o equivalentemente:

\[
\begin{equation}
|\tilde{x} - x| \leq \epsilon \label{2.8}\tag{5}
\end{equation}
\]

donde esperamos que \(\epsilon\) sea ``pequeño''. Tener acceso a tal
límite de error significa que podemos decir algo muy específico sobre el
(desconocido) valor exacto \(x\):

\[
\begin{equation}
\tilde{x} - \epsilon \leq x \leq \tilde{x} + \epsilon \label{2.9}\tag{6}
\end{equation}
\]

Esto significa que aunque no sepamos el valor exacto de \(x\), sabemos
que puede ser a lo más \(\tilde{x} + \epsilon\). Tenga en mente que si
usted sabe el concreto error absoluto, como con
\(\Delta x_0 = -10^{-3}\) en el ejemplo anterior, de la ecuación \_,
sabemos que

\[
\begin{equation}
x = \tilde{x} - \Delta x \label{2.10} \tag{7}
\end{equation}
\]

y no hay necesidad de desigualdades. Las desigualdades aparecen cuando
no se sabe el valor concreto del error absoluto y sólo se conoce el
límite para la magnitud del error. La notación para el límite de error
\(|\Delta x| \leq \epsilon\) a veces se reescribe en la forma
\(x = \tilde{x} \pm \epsilon\) aunque se tiene que tener cuidado: esto
utiliza nuestra definición de \emph{máximo de error} (es decir, el peor
caso posible) como se escribió anteriormente, no el \emph{error
estandar}\\
Por supuesto, incluso en esta pronta etapa, uno debe pensar en
exactamente a qué nos referimos por ``pequeño''. En el caso anterior de
\(\Delta x_0 = -10^{-3}\) probablemente se ajusta a esto. ¿Pero que tal:

\[
\begin{equation}
x_1 = 1 \ 000 \ 000 \ 000.0 \quad \quad \quad \tilde{x_1} = 999 \ 999 \ 999.0 \label{2.11}\tag{8}
\end{equation}
\]

Que corresponde a un error absoluto de \(\Delta x_1 = -1\)? Obviamente,
este error absoluto es más grande (en magnitud) que
\(\Delta x_0 = -10^{-3}\). Por otro lado, no es muy equivocado decir que
hay algo mal con esta comparación: \(x_1\) es mucho más grande (en
magnitud) que \(x_0\), por lo que aunque el valor aproximado
\(\tilde{x_1}\) está equivocado por una unidad, se ``siente'' más cerca
al valor exacto correspondiente de lo que \(\tilde{x_0}\) estaba.\\
Esto se resuelve introduciendo una nueva definición. Igual que antes,
estamos interesados en una cantidad cuyo valor exacto es \(x\) y un
valor aproximado para él es \(\tilde{x}\). Asumiento \(x \neq 0\),
podemos definir el \emph{error relativo} como sigue:

\[
\begin{equation}
\delta x = \frac{\Delta x}{x} = \frac{\tilde{x} - x}{x} \label{2.12} \tag{9}
\end{equation}
\] Obviamente, esto simplemente es el error absoluto \(\Delta x\)
dividido por el valor exacto \(x\). Al igual que antes, no especificamos
el origen de este error relativo (incertidumbre en los datos ingresados,
redondeo, etc.)\\
Otra forma de expresar este error relativo es:

\[
\begin{equation}
\tilde{x} = x(1 + \delta x) \label{2.13}\tag{10}
\end{equation}
\]

Usted debe convencerse de que esto sigue directamente de \eqref{2.12}.
En lo que sigue se usará esta formulación en repetidas ocasiones.\\
Aplicando esta definición del error relativo a los ejemplos anteriores:

\[
\delta x_0 = \frac{0.999 - 1.000}{1.000} = -10^{-3}, \ \ \ \ \ \ \ \ \ \ \ \ \ \ \
\delta x_1 = \frac{999 \ 999 \ 999.0 - 1 \ 000 \ 000 \ 000.0}{1 \ 000 \ 000 \ 000.0} = -10^{-9} \label{2.14}\tag{11}
\]

La definición de error relativo es consistente con nuestra intuición:
\(\tilde{x}_1\) es, en realidad, un mucho mejor estimado de \(x_1\) que
\(\tilde{x}_0\) es de \(x_0\). Con frecuencia. el error relativo es un
porcentaje dado: \(\delta x_0\) es un error relativo de \(-0.1\%\)
mientras que \(\delta x_1\) es un erro relativo de \(-10^{-7}\%\).\\
En física los valores de un observable pueden variar por varios órdenes
de magnitud (de acuerdo a la densidad, temperatura y demás), por lo que
es bueno usar, cuando sea posible, el concepto de error relativo, que es
independiente de la escala. Al igual que en el caso del error absoluto,
también se puede introducir un \emph{límite de error relativo}:

\[
\begin{equation}
|\delta x| = \left| \frac{\Delta x}{x} \right| \leq \epsilon \label{2.15}\tag{12}
\end{equation}
\]

donde ahora la frase ``\(\epsilon\) es pequeño'' no es ambigua.
Finalmente, note que la definición de error relativo en \eqref{2.12}
involucra a \(x\) en el denominador. Si se tiene acceso al valor exacto
(como en los ejemplos de arriba con \(x_0\) y \(x_1\)), todo está bien.
Pero si no se conoce este valor es más conveniente usar el valor
aproximado de \(\tilde{x}\) en el denominador.

\hypertarget{propagaciuxf3n-de-error}{%
\subsection{Propagación de error}\label{propagaciuxf3n-de-error}}

Hasta ahora, hemos examinado los conceptos de error absoluto y error
relativo (así como sus límites respectivos). Estos se discuten en
general, sin brindar detalles sobre las operaciones matemáticas donde se
utilizan estos números. Ahora procedemos a discutir las operaciones
elementales (suma, resta, multiplicación, división), que se espera le de
a usted conocimientos para combinar valores aproximados. Uno de nuestros
objetivos es ver qué pasa cuando juntamos límites de error de dos
números \(a\) y \(b\) para producir un límite de error de un tercer
número, \(x\). En otras palabras, estudiaremos la propagación de error.
En lo que sigue, es importante tener en mente que se usan \emph{errores
máximos}, por lo que los resultados son diferentes de los que se pueden
encontrar en un curso experimental de mediciones común.

\hypertarget{suma-o-resta}{%
\subsubsection{Suma o Resta}\label{suma-o-resta}}

Tenemos dos números reales, \(a\) y \(b\), y deseamos tomar su
diferencia:

\[
\begin{equation}
x = a - b \label{2.16}\tag{13}
\end{equation}
\]

Como es común, no sabemos los valores exactos, sólo sus aproximados
\(\tilde{a}\) y \(tilde{b}\), por lo que en su lugar formamos la
diferencia de estos:

\[
\begin{equation}
\tilde{x} = \tilde{a} - \tilde{b} \label{2.17}\tag{14}
\end{equation}
\]

Si ahora aplicamos la ecuación \eqref{2.10} dos veces:

\[
\begin{equation}
\tilde{a} = a + \Delta a, \ \ \ \ \ \ \ \ \ \ \tilde{b} = b + \Delta b \label{2.18}\tag{15}
\end{equation}
\]

Agregando las últimas cuatro ecuaciones en la definición de error
absoluto \eqref{2.5}, tenemos:

\[
\begin{equation}
\Delta x = \tilde{x} - x = (a + \Delta a) - (b + \Delta b) - (a - b) = \Delta a - \Delta b \label{2.19}\tag{16}
\end{equation}
\]

En la tercera igualdad cancelamos lo que podemos.\\
Ahora recordemos que estamos interesados en encontrar relaciones entre
límites de error. Por lo que tomamos el valor absoluto y luego usamos la
desigualdad triangular para encontrar:

\[
\begin{equation}
|\Delta x | \leq |\Delta a| + |\Delta b| \label{2.20}\tag{17}
\end{equation}
\]

Una derivación completamente análoga lleva al mismo resultado para el
caso de la suma de dos números \(a\) y \(b\). Por lo tanto, la
conclusión principal es que \emph{en la suma y resta sumar los límites
para los errores absolutos en los dos números nos da como resultado un
límite para el error absoluto}\\
Veamos un ejemplo. Asumamos que tenemos:

\[
\begin{equation}
|4.56 - a| \leq 0.14, \ \ \ \ \ \ \ \ \ \ |1.23 - b| \leq 0.03 \label{2.21}\tag{18}
\end{equation}
\]

(Si está confundido con esta notación, revise las ecuaciones \eqref{2.7}
y \eqref{2.8}). El hallazgo en \eqref{2.20} implica que la siguiente
relación se mantiene, cuando \(x = a - b\):

\[
\begin{equation}
|3.33 - x| \leq 0.17 \label{2.22}\tag{19}
\end{equation}
\]

Es fácil ver que este límite de error, simplemente la suma de los dos
límites de error con los que empezamos, es mayor que cualquiera de
ellos. Si no tuvieramos acceso a \eqref{2.20}, podríamos haber llegado
al mismo resultado por el camino largo: \textbf{(a)} cuando \(a\) tiene
el valor más grande posible (4.70) y \(b\) tiene el valor más pequeño
posible (1.20), se tiene el valor más grande posible para \(a - b\), que
es 3.50, y \textbf{(b)} cuando \(a\) tiene el valor más pequeño posible
(4.42) y \(b\) tiene el valor más grande posible (1.26), se obtiene el
valor más pequeño posible para \(a - b\), que es 3.16.\\
En la forma de el resultado principal \eqref{2.20}, que se aplicó a un
ejemplo específico, se muestra que simplemente se sumaron los límites de
error absoluto. Como se mencionó antes, esto es diferente de lo que se
hace cuando se enfrenta con ``errores estandar'' (como deviación
estandar de la distribución de muestras): en ese caso, los errores
absolutos agregan ``cuadratura''.

\hypertarget{cancelaciuxf3n-catastruxf3fica}{%
\subsubsection{Cancelación
catastrófica}\label{cancelaciuxf3n-catastruxf3fica}}

Examinemos un caso más interesante: \(a \approx b\) (para el que
\(x = a - b\) es pequeño). Dividiendo el resultado de \eqref{2.20} con
\(x\) nos da el error relativo (el límite) en \(x\):

\[
\begin{equation}
|\delta x| = \left| \frac{\Delta x}{x} \right| \leq \frac{|\Delta a| + |\Delta b|}{|a - b|} \label{2.23}\tag{20}
\end{equation}
\]

Ahora, expresando \(\Delta a\) y \(\Delta b\) en términos del error
relativo correspondiente: \(\Delta a = a \delta a\) y
\(\Delta b = b \delta b\). Como \(a \approx b\), se puede factorizar
\(|a|\):

\[
\begin{equation}
|\delta x| \leq (|\delta a| + |\delta b|)\frac{|a|}{|a - b|} \label{2.24}\tag{21}
\end{equation}
\]

Es facil ver que si \(a \approx b\) entonces \(|a - b|\) será mucho
menor que \(|a|\) por lo que, como la fracción será más grande, los
errores relativos \(\delta a\) y \(\delta b\) aumentaran{[}\^{}1{]}.
{[}\^{}1{]}: Este problema específico no ocurre en el caso de la suma,
ya que allí el denominador no tiene que ser pequeño

Hagamos un ejemplo. Supongamos que tenemos:

\[
\begin{equation}
|1.25 - a| \leq 0.03, \ \ \ \ \ \ \ \ \ \ |1.20 - b| \leq 0.03 \label{2.25}\tag{22}
\end{equation}
\]

es decir, un error relativo (límite)
\(\delta a \approx 0.03/1.25 = 0.024\) o más o menos \(2.4\%\) (esto es
un aproximado, porque se dividió con \(\tilde{a}\), no con el,
desconocido, \(a\)). Similarmente, el otro error relativo (límite) es
\(\delta b \approx 0.03/1.20\) o más o menos \(2.5\%\). De la ecuación
\eqref{2.24} vemos que el error relativo para la diferencia obedece:

\[
\begin{equation}
|\delta x| \leq (0.024 + 0.025) \frac{1.25}{0.05} = 1.225 \label{2.26}\tag{23}
\end{equation}
\]

donde el lado derecho es una aproximación (usando
\(\tilde{a} y \tilde{x}\)). Esto muestra que dos números con cerca de
\(2.5\%\) de errores relativos fueron restados y el resultado ¡tiene un
error relativo de mucho más del cien porciento! A esto a veces se le
llama \emph{cancelación catastrófica o restada}

\hypertarget{multiplicaciuxf3n-o-divisiuxf3n}{%
\subsubsection{Multiplicación o
división}\label{multiplicaciuxf3n-o-divisiuxf3n}}

Tenemos dos números reales \(a\) y \(b\), y deseamos tomar su producto:

\[
\begin{equation}
x = ab \label{2.27}\tag{24}
\end{equation}
\]

Como es usual, no sabemos sus valores exactos, sólo los valores
aproximados \(\tilde{a}\) y \(\tilde{b}\)

\hypertarget{propagaciuxf3n-de-error-general-funciones-de-una-variable}{%
\subsubsection{Propagación de error general: Funciones de una
variable}\label{propagaciuxf3n-de-error-general-funciones-de-una-variable}}

Lorem ipsum dolor sit amet, consectetur adipiscing elit. Pellentesque
fermentum dolor odio, quis auctor lectus dapibus nec. Praesent sagittis
eu nisi vitae placerat. Sed suscipit, augue ac blandit gravida, sem
felis pretium velit, quis aliquet urna odio sed nisl. Nam nibh purus,
dapibus sit amet convallis in, euismod a diam. Donec justo lacus, dictum
at elit ac, tincidunt facilisis mauris. Fusce non lorem et mi
ullamcorper pharetra. Orci varius natoque penatibus et magnis dis
parturient montes, nascetur ridiculus mus. Nulla blandit, turpis a
sagittis efficitur, magna est interdum arcu, at commodo nisl lorem id
libero. Etiam accumsan nulla vitae luctus consequat.

Donec laoreet vulputate risus eu sagittis. In malesuada a nunc ac
eleifend. Ut ut aliquam odio. Vestibulum suscipit, enim at pretium
interdum, ipsum tellus commodo orci, id placerat eros tellus a magna.
Maecenas nunc nisi, congue mattis est gravida, placerat condimentum
nibh. Aliquam malesuada blandit ipsum, in imperdiet arcu sagittis ut.
Fusce ex neque, viverra sit amet pellentesque vel, rutrum vel elit.
Etiam mi nisl, pulvinar vel lacus eget, fringilla ultricies nisl. Nulla
velit turpis, viverra nec elementum quis, mollis ut augue. Integer
vulputate, metus id cursus imperdiet, arcu nibh consequat justo, a porta
eros massa vel nibh. Vivamus sed nibh sed nunc pellentesque pharetra.
Etiam nec mi vel felis imperdiet interdum. Curabitur tristique dapibus
risus, eu tincidunt dui dapibus in. Donec tempor varius elit ut
vestibulum. Mauris metus libero, sollicitudin id nisi ut, cursus
tincidunt turpis. Aenean vitae placerat diam.

\hypertarget{propagaciuxf3n-de-error-general-funciones-de-varias-variables}{%
\subsubsection{Propagación de error general: Funciones de varias
variables}\label{propagaciuxf3n-de-error-general-funciones-de-varias-variables}}

Lorem ipsum dolor sit amet, consectetur adipiscing elit. Pellentesque
fermentum dolor odio, quis auctor lectus dapibus nec. Praesent sagittis
eu nisi vitae placerat. Sed suscipit, augue ac blandit gravida, sem
felis pretium velit, quis aliquet urna odio sed nisl. Nam nibh purus,
dapibus sit amet convallis in, euismod a diam. Donec justo lacus, dictum
at elit ac, tincidunt facilisis mauris. Fusce non lorem et mi
ullamcorper pharetra. Orci varius natoque penatibus et magnis dis
parturient montes, nascetur ridiculus mus. Nulla blandit, turpis a
sagittis efficitur, magna est interdum arcu, at commodo nisl lorem id
libero. Etiam accumsan nulla vitae luctus consequat.

Donec laoreet vulputate risus eu sagittis. In malesuada a nunc ac
eleifend. Ut ut aliquam odio. Vestibulum suscipit, enim at pretium
interdum, ipsum tellus commodo orci, id placerat eros tellus a magna.
Maecenas nunc nisi, congue mattis est gravida, placerat condimentum
nibh. Aliquam malesuada blandit ipsum, in imperdiet arcu sagittis ut.
Fusce ex neque, viverra sit amet pellentesque vel, rutrum vel elit.
Etiam mi nisl, pulvinar vel lacus eget, fringilla ultricies nisl. Nulla
velit turpis, viverra nec elementum quis, mollis ut augue. Integer
vulputate, metus id cursus imperdiet, arcu nibh consequat justo, a porta
eros massa vel nibh. Vivamus sed nibh sed nunc pellentesque pharetra.
Etiam nec mi vel felis imperdiet interdum. Curabitur tristique dapibus
risus, eu tincidunt dui dapibus in. Donec tempor varius elit ut
vestibulum. Mauris metus libero, sollicitudin id nisi ut, cursus
tincidunt turpis. Aenean vitae placerat diam.

    \hypertarget{matrices}{%
\section{Matrices}\label{matrices}}

\hypertarget{ejemplos-de-fuxedsica}{%
\subsection{Ejemplos de física}\label{ejemplos-de-fuxedsica}}

En esta subsección se discutirán algunos ejemplos elementales de física
de pregrado, que no involucran cálculos pesados, pero sí involucran los
mismos conceptos.

\begin{enumerate}
\def\labelenumi{\arabic{enumi}.}
\tightlist
\item
  Rotaciones en dos dimensiones\\
  Considere un sistema coordenado de dos dimenciones. Un punto
  \(r = (x \ y)^T\) puede ser rotado en contra de las manecillas del
  reloj a través de un ángulo \(\theta\) en el origen, produciendo un
  nuevo punto \(r' = (x' \ y')^T\). Las coordenadas de los dos puntos
  están relacionadas como sigue:
\end{enumerate}

\[
\begin{pmatrix}
\cos \theta & -\sin \theta \\
\sin \theta & \cos \theta
\end{pmatrix}
\begin{pmatrix}
x \\ y
\end{pmatrix} 
=
\begin{pmatrix}
x' \\ y'
\end{pmatrix}
\]

    \begin{tcolorbox}[breakable, size=fbox, boxrule=1pt, pad at break*=1mm,colback=cellbackground, colframe=cellborder]
\prompt{In}{incolor}{4}{\hspace{4pt}}
\begin{Verbatim}[commandchars=\\\{\}]
\PY{n+nb}{print}\PY{p}{(}\PY{l+s+s2}{\PYZdq{}}\PY{l+s+s2}{Hello World!}\PY{l+s+s2}{\PYZdq{}}\PY{p}{)}
\PY{n+nb}{print}\PY{p}{(}\PY{l+m+mi}{1}\PY{o}{+}\PY{l+m+mi}{2}\PY{p}{)}
\end{Verbatim}
\end{tcolorbox}

    \begin{Verbatim}[commandchars=\\\{\}]
Hello World!
3
\end{Verbatim}


    % Add a bibliography block to the postdoc
    
    
    
    \end{document}
